\chapter*{Введение}

Сколько я себя помню – всегда хотел написать книжку. В детстве мне казалось, что человек, который читает много книг, просто обязан уметь их писать. Это занятие я считал
настолько лёгким и непринуждённым, что от собственно написания книги меня всегда отделяла какая-то малость – домашнее задание в школе, например. Так что графоманом я
был неправильным – книжек не писал и не пытался, только думал о том, что когда-нибудь было бы так здорово...

Для тех, кому стало неинтересно читать эту главу уже сейчас, после первого абзаца, сообщаю – продолжение разговора про Maya будет уже скоро, всего через полсотни
страниц.

Для всех же остальных – после небольшой паузы, пока самые нетерпеливые листают в поисках начала новой главы – сообщаю: меня зовут Алексей Пузиков, и моя мечта сбылась.
Частично сбылась, конечно – пишу я не целую книгу, а только эту главу. И говорить в ней мы будем о внешних рендерерах, их связи с Maya и о Renderman.

Итак, что же пропустят наши неугомонные друзья?
\begin{itemize}
\item рассказ о том, зачем Maya нужны внешние рендереры
\item и как они работают в связке
\item а ещё про Renderman вообще
\item и про Photorealistic Renderman в частности
\item и про MTOR
\item и даже немножко про Gelato, Jot, GRUNT, MayaMan, Liquid и прочие вкусности, названия и аббревиатуры
\item и наконец, на десерт, они не прочтут краткое пособие по новому экстремальному виду спорта – Renderman-читингу.
\end{itemize}

Уже интересно? Ну что ж, давайте начнём!

Однако прежде всего – раз уж мы так весело начали и замахнулись на такой объём информации – поступим с читателем честно и укажем также, чего вы в этой главе не
найдёте. А не найдёте вы в ней подробных учебников, упражнений и уроков, справочных руководств и перевода спецификаций. Мир внешних рендереров необъятен, да что там
говорить – про один Renderman можно (и нужно!) написать не одну толстенную книжку – а у нас с вами в распоряжении всего лишь небольшая глава. Так что нам придётся
ограничиться неким набором ключевых моментов, опираясь на которые вы сможете самостоятельно продолжить изучение этого нового мира и выйти на качественно новый уровень в
своей работе.

% \gr{{chapterRendermanOK_files/image003}
