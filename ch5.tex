\chapter*{Ужасная история в поезде}
 

Тихий зимний вечер. Мы вместе с классом возращаемся
   из поездки в Прибалтику. Поезд проносится мимо тёмных полей и
   лесов. Парни внизу играют в карты, на полке напротив тоже что-то
   происходит – но мне всё равно, я читаю. Мне только что дали
   почитать СуперПовесть.
 

Я чуть ли не единственный участник поездки, который
   ещё не успел прочитать эту повесть. Она хит сезона, она ходит по
   рукам, от неё невозможно оторвать, очередь выстроилась на неделю
   вперёд. Класс разделился на тех, кто прочитал Произведение - и со
   смаком обсуждает перипетии сюжетных поворотов - и тех, кто ещё не
   успел и может только тупо кивать, слушая эти обсуждения.
 

Страницы уже немного замусолены, этот десяток
   листков был выдран из какого-то новомодного журнала для молодёжи,
   который уже не боится печатать “про секс”, но всё ещё не научился
   правильно пользоваться компьютерной вёрсткой.
 

И вот оно, счастье – хозяин манускрипта едет со
   мной в одном купе – или это всё-таки был плацкарт? Память услужливо
   подсовывает мне любые детали, кроме этой. Я пользуюсь моментом – и
   встреваю без очереди, выпросив эти пожмаканые листики у хозяина и
   клятвенно пообещав, что быстренько прочитаю и верну.
 

О, что это была за повесть! Отложив в сторону её
   содержание (а именно, борьбу бравого зелёного берета за выживание в
   диких джуглях Амазонии), стоит отметить исключительно занятный
   художественный стиль. Ломаное повествование всё время бросало меня
   из одного ключевого момента в другой. Казалось, что в произведении
   совсем нет сюжетной линии, и лишь к тому моменту, когда я осилил
   примерно половину листков – скрючившись под еле светящейся
   лампочкой на второй полке ПЛАЦКАРТА, Я ВСПОМНИЛ, уже далеко за
   полночь, безумно уставший, борясь со сном, со слипающимися и
   немного слезящимися глазами, под монотонный стук колёс и
   покачивание вагона разогнавшегося поезда – я наконец-то понял, что
   произведение построено наоборот – сюжетная линия ведёт меня от
   конца к началу повествования, время от времени подбрасывая новые
   заморочки в своём движении к финалу апофеоза.
 

Я уже говорил, что на листиках не было номеров
   страниц?
 

Всю чудовищность произошедшего я осознал лишь в тот
   момент, когда увидел на последней странице ЗАГОЛОВОК повести. Как
   оказалось, всё это время я боролся с сюжетом и плохим освещением,
   читая повесть наоборот – от последнего листика к
   первому.
 

К чему бы я городил воспоминания школьных лет в
   книжке про Maya, спросите вы? Да просто потому, что, по моему
   искреннему мнению, начинать рассказ о Prman с описания
   функциональности Renderman Artist Tools (RAT) – это всё равно, что
   начинать читать интереснейшую книжку с конца. Нет, не так – это всё
   равно, что начинать читать учебник с конца. Это антипедагогично,
   антинаучно, антиморально и вообще неправильно.
 

И именно это мы собираемся сделать сейчас – начать рассказ с RATа.
 

Конечно, можно было бы всё-таки послушать зов
   своего сердца и сделать из этой главы академический учебник –
   начать с азов, постепенно продвигаясь ко всё более и более сложным
   вещам. Но так уж получилось, что самое интересное и самое
   презентабельно выглядящее в prman – это 2 большие разницы. Да и, в
   конце концов, мы же не ставим себе целью усыпить нашу аудиторию
   посреди главы?
 

Поэтому придётся поступиться принципами и построить
   обзор возможностей комбинации prman+Maya в точности так, как я
   когда-то читал ту самую повесть – то есть задом наперёд. Сначала мы
   с вами рассмотрим RAT, входяшие в него утилиты, их особенности и
   алгоритм работы с ними. Затем пойдём назад (или всё-таки вперёд?),
   опустимся чуть поглубже в потроха рендерера и узнаем, как оно всё
   там внутри работает и вызывается.
 

Кстати говоря – несмотря на то самое происшествие в
   поезде, я всё ещё не могу избавиться от дурной привычки читать с
   конца журналы. Хотя в последнее время поступаю так всё реже и
   реже.