\chapter*{Альтернативы?}
 

Почему же множество людей, FX-домов и студий во
   всём мире НЕ используют
   Renderman Pro Server? Тому есть множество причин.
 

Начнём с очевидной - цены.
 

На этом шаге можно было бы сразу и закончить,
   потому что по цене комплект из Renderman Pro Server, RAT + годичная
   подписка на услуги службы поддержки почти догнал (чуть было не
   написал – автомобиль) Maya Unlimited – надеюсь, вы оценили. Один
   этот факт ставит жирный крест на продукте для маленьких студий,
   фрилансеров и студентов, которые хотят поэкспериментировать с
   рендерером дома – они выбирают либо совместимые альтернативы, либо
   совершенно другие продукты.
 

Но я хотел заострить ваше внимание на другом
   аспекте проблемы. Да, prman – очень универсальный рендерер. Но,
   согласно поговорке, когда вы берёте в руки молоток, то все предметы
   вокруг вас начинают казаться гвоздями.
 

И поэтому мы сделаем небольшой шаг в сторону от
   нашего магистрального направления (“Maya+Renderman=счастье”) и
   попытаемся объять необъятное – рассмотреть альтернативные prman’у
   рендереры, которые можно использовать совместно с Maya.
 

Вообще говоря, говорить о том, что какой-то
   рендерер не совместим с Maya, особенно после того, как вы
   познакомились с главой, посвященной Mel – глупо. Возможности Mel в
   области вывода данных из Maya настолько велики, что фактически любой рендерер,
   запускаемый из командной строки, можно с теми или иными затратами
   прикрутить к Майе. И это будет работать – а если вас перестанет
   устраивать скорость работы скрипта - то вы просто перепишете ваш
   скрипт в виде плагина (или попросите кого-то переписать ваш скрипт
   в виде плагина). А это означает, что Maya-совместимым является
   почти любой из существующих на рынке рендереров, запускаемых из
   командной строки. Но, как говорил Козьма Прутков, нельзя объять
   необъятное – у нас здесь не Большая Советская Энциклопедия, в конце
   концов. И поэтому мы сделаем небольшую выборку из огромного списка
   и расскажем вам о нескольких внешних рендерерах.

 \section*{Какие рендереры?}
 

Как-то вечером я задался вопросом – какие вообще
   рендереры существуют и почему они существуют вообще? Вопрос
   настолько же риторический, насколько и философский – по аналогии,
   можно спросить, почему так много моделей лопат или молотков есть в
   магазине.
 

Во-первых, очень часто большая (а иногда – и не
   очень большая) студия самостоятельно изготавливает для себя рабочий
   инструментарий – начиная с рендерера и заканчивая системами
   моделинга, анимации, цветокоррекции и композитинга. В таком случае
   сотрудники этой студии получают максимальный контроль над
   результатами своего труда.
 

Во-вторых, очень часто существующие решения не
   справляются с поставленными для них задачами, неважно, из-за своих
   особенностей или из-за особенностей таких задач. Для таких задач
   (типичные примеры – аниме, волосы, жидкость и пламя) достаточно
   часто пишутся специальные рендереры – которые умеют считать только
   один вид геометрии или один спецэффект – но делают его хорошо и
   очень быстро.
 

Ну и на закуску остаются – исследовательские и
   студенческие проекты, источник вдохновения и исходного кода, поток
   новых идей и инноваций.
 

Чем же закончился тот вечер, когда я попытался
   объять необъятное? Простой схемой, на которой я перечислил все
   известные мне (пусть только по названию) рендереры и провёл между
   некоторыми из них связи – будь то родственные, технологические или
   какие-то ещё. Картинка перед вами. 

\gr{image063}
 

Тёмно-серым цветом на этой схеме обозначены некие
   ключевые продукты, от которых я отталкивался в выстраивании системы
   отсчёта – краеугольные камни. Три таких камня очевидны – это
   Renderman. Mental Ray и видеокарты (OpenGL/DirectX и прочее). Но
   есть и четвёртый камень в этой схеме, и этот четвертый элемент -
   Gelato.

