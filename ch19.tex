\chapter*{Думайте о неожиданном}
  

 Экспериментируйте. Пробуйте. Применяйте новые
    интересные техники в необычных ситуациях. 
  

 Интернет битком набит информацией, новостями,
    туториалами. Почти каждый год на сайте renderman.org выкладываются
    новые PDFы с материалами конференции Siggraph и учебного курса по
    Renderman, проходящего на этой конференции. Эти документы – кладезь, неисчерпаемый
    поток знаний, умений и опыта, концентрированное знание множества
    очень умных людей, которые оттачивают все эти приёмы на
    каждодневной практике реального кино- и видео-производства. Их
    можно читать наискосок, проходить по шагам, постоянно сверяясь с
    текстом, пытаться шаблонно использовать в своей работе – а можно
    подойти творчески и начать пробовать и задавать вопросы.
  

 Что будет, если в шейдере в diffuse подмешать
    specular?
  

 А если использовать diffuse в качестве
    specular?
  

 Что будет, если на вход шейдера Blinn повесить
    выход шейдера Noise?
  

 А что получится, если в одном шейдере использовать
    несколько различных спекуляров?
  

 Можно ли использовать карту теней для ускорения
    просчёта сцены (ответ: можно)?
  

 Что получится, если выставить источнику света
    отрицательную величину яркости?
  

 А если связать источник с текстурой, которая бы
    моделировала освещение из окна – понадобится ли само
    окно?
  

 А можно ли сделать такой свет, который был бы синим
    в тени, и белым в остальных областях? А чтобы он содержал только
    specular-составляющую? А чтобы он светил в одну сторону, а тени
    отбрасывал – в другую? А чтобы он мог отбрасывать тени без объекта,
    сам?
  

 Что вы скажете по поводу анимированных текстур,
    которые содержат в себе текстурные координаты другого
    объекта?
  

 А как вам идея насчёт двух карт глубины – одной из
    камеры, а другой из источника света – и определения разницы между
    их значениями?
  

 Можно ли написать шейдер, который покажет, в каком
    месте кривизна поверхности максимальна?
  

 Можно ли хранить в текстуре массив – и потом
    обращаться к нему по индексу?
  

 Можно ли написать шейдер, который в случае ошибки
    отправит вам SMS?
  

 Как известно, наука – это удовлетворение
    собственного любопытства за счёт заказчика. Так вот Chi-Ting – это
    наука. Удовлетворите своё любопытство. Получите удовольствие от
    своих открытий. Поделитесь ними с окружающими.
  

 Кинопроектор показывает 24 кадра в секунду;
    телеэкран – 25. Объем информации, сваливающийся на зрителя,
    настолько велик, что если вы его немного обманете, но он не заметит
    подвоха и поверит в реальность происходящего – то вы
    победили.